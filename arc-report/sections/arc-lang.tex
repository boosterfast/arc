\section{Arc-Lang}

% Introduction to what Arc-Lang is
Arc-Lang is a language for data analytics. 

\subsection{Tour of Arc-Lang}

% Highlight each of Arc-Lang's novel features
% * Streams
% * Frames
% * Tensors
% * Tasks
This section highlights the main features of the Arc-Lang.

% \lstinputlisting[linerange=example-example]{../arc-lang/examples/assign.arc}

\subsection{Features}

% * Functional Features (exposed to the user)
%   - Values, types, and constructors and destructors
%     - Statically vs Dynamically sized types
%     - Type constraints
%     - Pass by value
%     - Difference between "Big Data" and "Small Data"
%   - Controlflow
%   - Dataflow
%   - Declarative programming
%   - Imperative programming
%   - Concurrency
%   - Workflow
%   - Polymorphism
% * Non-Functional Features (not exposed to the user)
%   - Parallelism
% * Go into the details of each feature.
% * Write about possible alternative designs.
% * Give examples for each feature
